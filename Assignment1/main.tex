%iffalse
\let\negmedspace\undefined
\let\negthickspace\undefined
\documentclass[journal,12pt,twocolumn]{IEEEtran}
\usepackage{cite}
\usepackage{amsmath,amssymb,amsfonts,amsthm}
\usepackage{algorithmic}
\usepackage{graphicx}
\usepackage{textcomp}
\usepackage{xcolor}
\usepackage{txfonts}
\usepackage{listings}
\usepackage{enumitem}
\usepackage{mathtools}
\usepackage{gensymb}
\usepackage{comment}
\usepackage[breaklinks=true]{hyperref}
\usepackage{tkz-euclide} 
\usepackage{listings}
\usepackage{gvv}                                        
%\def\inputGnumericTable{}                                 
\usepackage[latin1]{inputenc}                                
\usepackage{color}                                            
\usepackage{array}                                            
\usepackage{longtable}                                       
\usepackage{calc}                                             
\usepackage{multirow}                                         
\usepackage{hhline}                                           
\usepackage{ifthen}                                           
\usepackage{lscape}
\usepackage{tabularx}
\usepackage{array}
\usepackage{float}

\usepackage{multicol}

\newtheorem{theorem}{Theorem}[section]
\newtheorem{problem}{Problem}
\newtheorem{proposition}{Proposition}[section]
\newtheorem{lemma}{Lemma}[section]
\newtheorem{corollary}[theorem]{Corollary}
\newtheorem{example}{Example}[section]
\newtheorem{definition}[problem]{Definition}
\newcommand{\BEQA}{\begin{eqnarray}}
\newcommand{\EEQA}{\end{eqnarray}}
\newcommand{\define}{\stackrel{\triangle}{=}}
\theoremstyle{remark}
\newtheorem{rem}{Remark}

% Marks the beginning of the document
\begin{document}
\bibliographystyle{IEEEtran}
\vspace{3cm}

\title{Assignment 1 - EE1030}
\author{ee24btech11018 - D. Swaraj Sharma}
\maketitle
\newpage
\bigskip

\renewcommand{\thefigure}{\theenumi}
\renewcommand{\thetable}{\theenumi}

\section{\textbf{Section-B} // \textbf{JEE Main} / \textbf{AIEEE}}

\begin{enumerate}[label={\arabic*.}]
%minor edit

\item {If $ 1, \log_9 \brak{3^{1-x} +2}, \log_3 \brak{4\cdot3^x -1}$ are in A.P then $x$ equals}
{\hfill{\sbrak{2002}}}
\begin{enumerate}
\begin{multicols}{2}
\item  {$\log_3 4$}
 \item {$1-\log_3 4$}
 \item {$1-\log_4 3$}
 \item {$\log_4 3$}
\end{multicols}
\end{enumerate}
\item {$l, m, n$ are the $p^{th}, q^{th}$ and $r^{th}$ term of a G.P. all positive, then $\mydet{\log l & p & 1 \\ \log m & q & 1 \\ \log n & r & 1 }$ equals}
{\hfill{\sbrak{2002}}} 
\begin{enumerate}
\begin{multicols}{2}
\item{$1$}
\item{$2$}
\item{$1$}
\item{$0$}
\end{multicols}
\end{enumerate}
% minor edit
\item {The value of $2^{\frac{1}{4}}\cdot 4^{\frac{1}{8}}\cdot 8^{\frac{1}{16}} \ldots \infty$ is}
{\hfill{\sbrak{2002}}} 
\begin{enumerate}
\begin{multicols}{2}
\item  {$1$}
\item  {$2$}
\item  {$\frac{3}{2}$}
\item  {$4$}
\end{multicols}
\end{enumerate}
\item {Fifth term of a GP is $2$, then the product of its $9$ terms is}
{\hfill{\sbrak{2002}}}
\begin{enumerate}	
\begin{multicols}{2}
\item  {$256$}
\item  {$512$}
\item  {$1024$}
\item  {none of these}
		\end{multicols}
\end{enumerate}

\item {Sum of infinite number of terms of a GP is $20$ and sum of their square is $100$. The common ratio of GP is}
{\hfill{\sbrak{2002}}} 
\begin{enumerate}
\begin{multicols}{2}

\item  {$5$}
\item  {$\frac{3}{5}$}
\item  {$\frac{8}{5}$}
\item  {$\frac{1}{5}$}
\end{multicols}
\end{enumerate} 

\item {$1^{3}-2^{3}+3^{3}-4^{3}+...
+9^{3}=$}
{\hfill{\sbrak{2002}}} 
\begin{enumerate}
\begin{multicols}{2}
\item  {$425$}
\item  {$-425$}
\item  {$475$}
\item  {$-475$}
\end{multicols}
\end{enumerate}

\item {The sum of the series \\ $\frac{1}{1\cdot2}-\frac{1}{2\cdot3}+\frac{1}{3\cdot4}\cdots \text{ up to } \infty$ is equal to} 
{\hfill{\sbrak{2003}}} 
\begin{enumerate}
\begin{multicols}{2}
\item  {$\log_e \brak{\frac{4}{e}}$}
\item  {2$\log_e 2$}
\item  {$\log_e 2-1$}
\item  {$\log_e 2$}
\end{multicols}
\end{enumerate}

\item {If $S_n = \sum\limits_{r=0}^{n} \frac{1}{\comb{n}{r}}$ and $t_n = \sum\limits_{r=0}^{n} \frac{r}{\comb{n}{r}}$, then $\frac{t_n}{S_n}$ is equal to}
{\hfill{\sbrak{2004}}} 
\begin{enumerate}
\begin{multicols}{2}
\item  {$\frac{2n-1}{2}$}
\item  {$\frac{1}{2}n-1$}
\item  {$n-1$}
\item  {$\frac{1}{2}n$}
\end{multicols}
\end{enumerate}

\item {Let $T_r$ be the $r^{th}$ term of an A.P. whose first term is a and common difference is $d$. If for some positive integers $m,n, m\neq n, T_m = \frac{1}{n}$ and $T_n = \frac{1}{m}$, then $a-d$ equals} 
{\hfill{\sbrak{2004}}}
\begin{enumerate}
\begin{multicols}{2}
\item  {$\frac{1}{m}+\frac{1}{n}$}
\item  {$1$}
\item  {$\frac{1}{mn}$}
\item  {$0$}
\end{multicols}
\end{enumerate}

\item {The sum of the first $n$ terms of the series $1^2+2\cdot2^2+3^2+2\cdot4^2+5^2+2\cdot6^2+\cdots$ is $\frac{n\brak{n+1}^2}{2}$ when $n$ is even. When $n$ is odd the sum is}
{\hfill{\sbrak{2004}}}
\begin{enumerate}
\begin{multicols}{2}
\item  {$\sbrak{\frac{n\brak{n+1}}{2}}^2$}
\item  {$\frac{n^2\brak{n+1}}{2}$}
\item  {$\frac{n\brak{n+1}^2}{4}$}
\item  {$\frac{3n\brak{n+1}}{2}$}
\end{multicols}
\end{enumerate}

\item {The sum of series $\frac{1}{2!}+\frac{1}{4!}+\frac{1}{6!}+\cdots$ is}
{\hfill{\sbrak{2004}}} 
\begin{enumerate}
\begin{multicols}{2}

\item  {$\frac{\brak{e^2-2}}{e}$}
\item  {$\frac{n^2\brak{n+1}}{2}$}
\item  {$\frac{n\brak{n+1}^2}{2e}$}
\item  {$\frac{\brak{e^2-1}}{2}$}
\end{multicols}
\end{enumerate}

\item {If the coefficients of $r^{th}, \brak{r+1}^{th}$, and $\brak{r+2}^{th}$ terms in the bionomial expansion of $\brak{1+y}^m$ are in A.P., then $m$ and $r$ satisfy the equation}

{\hfill{\sbrak{2005}}} 
\begin{enumerate}
\item  {$m^2-m\brak{4r-1}+4r^2-2=0$}
\item  {$m^2-m\brak{4r+1}+4r^2+2=0$}
\item  {$m^2-m\brak{4r+1}+4r^2-2=0$}
\item  {$m^2-m\brak{4r-1}+4r^2+2=0$}
\end{enumerate}

\item {If $x$ = $\sum\limits_{n=0}^{\infty}a^n$, $y$ = $\sum\limits_{n=0}^{\infty}b^n$, $z$ = $\sum\limits_{n=0}^{\infty}c^n$ where $a,b,c$ are in A.P and $\abs{a}<1,\abs{b}<1,\abs{c}<1$ then $x,y,z$ are in}
{\hfill{\sbrak{2005}}} 
\begin{enumerate}
\item  {G.P.}
\item  {A.P.}
\item  {Arithmetic - Geometric Progression}
\item  {H.P.}
\end{enumerate}

\item {The sum of the series $1+\frac{1}{4\cdot2!}+\frac{1}{16\cdot4!}+\frac{1}{64\cdot6!}+\cdots  \infty$ is}
{\hfill{\sbrak{2005}}} 
\begin{enumerate}
\begin{multicols}{2}
\item  {$\frac{e-1}{\sqrt{e}}$}
\item  {$\frac{e+1}{\sqrt{e}}$}
\item  {$\frac{e-1}{2\sqrt{e}}$}
\item  {$\frac{e+1}{2\sqrt{e}}$}
\end{multicols}
\end{enumerate}

\item {Let $a_1, a_2, a_3 \cdots$ be terms on A.P. If $\frac{a_1+a_2+\cdots a_p}{a_1+a_2+\cdots a_q}= \frac{p^2}{q^2},p \neq q$, then $\frac{a_6}{a_{21}}$ equals}
{\hfill{\sbrak{2006}} }
\begin{enumerate}
\begin{multicols}{2}
\item  {$\frac{41}{11}$}
\item  {$\frac{7}{2}$}
\item  {$\frac{2}{7}$}
\item  {$\frac{11}{41}$}
\end{multicols}
\end{enumerate}
\end{enumerate}

\end{document}

%iffalse
\let\negmedspace\undefined
\let\negthickspace\undefined
\documentclass[journal,12pt,twocolumn]{IEEEtran}
\usepackage{cite}
\usepackage{amsmath,amssymb,amsfonts,amsthm}
\usepackage{algorithmic}
\usepackage{graphicx}
\usepackage{textcomp}
\usepackage{xcolor}
\usepackage{txfonts}
\usepackage{listings}
\usepackage{enumitem}
\usepackage{mathtools}
\usepackage{gensymb}
\usepackage{comment}
\usepackage[breaklinks=true]{hyperref}
\usepackage{tkz-euclide} 
\usepackage{listings}
\usepackage{gvv}                                        
%\def\inputGnumericTable{}                                 
\usepackage[latin1]{inputenc}                                
\usepackage{color}                                            
\usepackage{array}                                            
\usepackage{longtable}                                       
\usepackage{calc}                                             
\usepackage{multirow}                                         
\usepackage{hhline}                                           
\usepackage{ifthen}                                           
\usepackage{lscape}
\usepackage{tabularx}
\usepackage{array}
\usepackage{float}


\newtheorem{theorem}{Theorem}[section]
\newtheorem{problem}{Problem}
\newtheorem{proposition}{Proposition}[section]
\newtheorem{lemma}{Lemma}[section]
\newtheorem{corollary}[theorem]{Corollary}
\newtheorem{example}{Example}[section]
\newtheorem{definition}[problem]{Definition}
\newcommand{\BEQA}{\begin{eqnarray}}
\newcommand{\EEQA}{\end{eqnarray}}
\newcommand{\define}{\stackrel{\triangle}{=}}
\theoremstyle{remark}
\newtheorem{rem}{Remark}

% Marks the beginning of the document
\begin{document}
\bibliographystyle{IEEEtran}
\vspace{3cm}

\title{Assignment 1 - EE1030}
\author{ee24btech11018 - D. Swaraj Sharma}
\maketitle
\newpage
\bigskip

\renewcommand{\thefigure}{\theenumi}
\renewcommand{\thetable}{\theenumi}

\section*{\textbf{Section-B} // \textbf{JEE Main} / \textbf{AIEEE}}

\begin{enumerate}[label=\textcolor{magenta}{\arabic*.}]


\item {If $ 1, \log_9 (3^{1-x} +2), \log_3 (4.3^x -1$ are in A.P then $x$ equals}\\ \textcolor {magenta}{\hfill{[2002]}} \\ 
\begin{enumerate}[label={(\alph*)}]
\item  {$\log_1 4$}
 \item {$1-\log_3 4$}
 \item {$1-\log_4 3$}
 \item {$\log_4 3$}
\end{enumerate}
\item {$l, m, n$ are the $p^{th}, q^{th}$ and $r^{th}$ term of a G.P. all positive, then $\left|\begin{matrix} \log l & p & 1 \\ \log m & q & 1 \\ \log n & r & 1 \end{matrix}\right|$ equals}\\ \textcolor {magenta}{\hfill{[2002]}} \\
\begin{enumerate}[label={(\alph*)}]
\item{1}
\item{2}
\item{1}
\item{0}
\end{enumerate}

\item {The value of $2^{\frac{1}{4}} . 4^{\frac{1}{8}} . 8^{\frac{1}{16}} \ldots \infty$ is}\\ \textcolor {magenta}{\hfill{[2002]}} \\
\begin{enumerate}[label={(\alph*)}]
\item  {1}
\item  {2}
\item  {3/2}
\item  {4}
\end{enumerate}

\item {Fifth term of a GP is 2, then the product of its 9 terms is}\\ \textcolor {magenta}{\hfill{[2002]}} \\
\begin{enumerate}[label={(\alph*)}]
\item  {256}
\item  {512}
\item  {1024}
\item  {none of these}
\end{enumerate}

\item {Sum of infinite number of terms of a GP is 20 and sum of their square is 100. The common ratio of GP is}\\ \textcolor {magenta}{\hfill{[2002]}} \\
\begin{enumerate}[label={(\alph*)}]
\item  {5}
\item  {3/5}
\item  {8/5}
\item  {1/5}
\end{enumerate} 

\item {$1^{3}-2^{3}+3^{3}-4^{3}+...
+9^{3}=$}\\ \textcolor {magenta}{\hfill{[2002]}} \\
\begin{enumerate}[label={(\alph*)}]
\item  {425}
\item  {-425}
\item  {475}
\item  {-475}
\end{enumerate}

\item {The sum of the series \\ $\frac{1}{1.2}-\frac{1}{2.3}+\frac{1}{3.4}\cdots \text{ up to } \infty$ is equal to}\\ \textcolor {magenta}{\hfill{[2003]}} \\
\begin{enumerate}[label={(\alph*)}]
\item  {$\log_e (\frac{4}{e})$}
\item  {2$\log_e 2$}
\item  {$\log_e 2-1$}
\item  {$\log_e 2$}
\end{enumerate}

\item {If $S_n = \sum\limits_{r=0}^{n} \frac{1}{{}^{n}C_{r}}$ and $t_n = \sum\limits_{r=0}^{n} \frac{r}{{}^{n}C_{r}}$, then $\frac{t_n}{S_n}$ is equal to}\\ \textcolor {magenta}{\hfill{[2004]}} \\
\begin{enumerate}[label={(\alph*)}]
\item  {$\frac{2n-1}{2}$}
\item  {$\frac{1}{2}n-1$}
\item  {$n-1$}
\item  {$\frac{1}{2}n$}
\end{enumerate}

\item {Let $T_r$ be the rth term of an A.P. whose first term is a and common difference is $d$. If for some positive integers $m,n, m\neq n, T_m = \frac{1}{n}$ and $T_n = \frac{1}{m}$, then $a-d$ equals}\\ \textcolor {magenta}{\hfill{[2004]}} \\
\begin{enumerate}[label={(\alph*)}]
\item  {$\frac{1}{m}+\frac{1}{n}$}
\item  {1}
\item  {$\frac{1}{mn}$}
\item  {0}
\end{enumerate}

\item {The sum of the first $n$ terms of the series $1^2+2.2^2+3^2+2.4^2+5^2+2.6^2+\cdots$ is $\frac{n(n+1)^2}{2}$ when $n$ is even. When $n$ is odd the sum is}\\ \textcolor {magenta}{\hfill{[2004]}} \\
\begin{enumerate}[label={(\alph*)}]
\item  {$\left[\frac{n(n+1)}{2}\right]^2$}
\item  {$\frac{n^2(n+1)}{2}$}
\item  {$\frac{n(n+1^2}{4}$}
\item  {$\frac{3n(n+1)}{2}$}
\end{enumerate}

\item {The sum of series $\frac{1}{2!}+\frac{1}{4!}+\frac{1}{6!}+\cdots$ is}\\ \textcolor {magenta}{\hfill{[2004]}} \\
\begin{enumerate}[label={(\alph*)}]
\item  {$\frac{(e^2-2)}{e}$}
\item  {$\frac{n^2(n+1)}{2}$}
\item  {$\frac{n(n+1)^2}{2e}$}
\item  {$\frac{(e^2-1)}{2}$}
\end{enumerate}

\item {If the coefficients of rth, $(r+1)$th, and $(r+2)$th terms in the bionomial expansion of $(1+y)^m$ are in A.P., then $m$ and $r$ satisfy the equation\\} \\ \textcolor {magenta}{\hfill{[2005]}} \\
\begin{enumerate}[label={(\alph*)}]
\item  {$m^2-m(4r-1)+4r^2-2=0$}
\item  {$m^2-m(4r+1)+4r^2+2=0$}
\item  {$m^2-m(4r+1)+4r^2-2=0$}
\item  {$m^2-m(4r-1)+4r^2+2=0$}
\end{enumerate}

\item {If $x$ = $\sum\limits_{n=0}^{\infty}a^n$, $y$ = $\sum\limits_{n=0}^{\infty}b^n$, $z$ = $\sum\limits_{n=0}^{\infty}c^n$ where $a,b,c$ are in A.P and $|a|<1,|b|<1,|c|<1$ then $x,y,z$ are in}\\ \textcolor {magenta}{\hfill{[2005]}} \\
\begin{enumerate}[label={(\alph*)}]
\item  {G.P.}
\item  {A.P.}
\item  {Arithmetic - Geometric Progression}
\item  {H.P.}
\end{enumerate}

\item {The sum of the series $1+\frac{1}{4.2!}+\frac{1}{16.4!}+\frac{1}{64.6!}+\cdots \infty is$}\\ \textcolor {magenta}{\hfill{[2005]}} \\
\begin{enumerate}[label={(\alph*)}]
\item  {$\frac{e-1}{\sqrt{e}}$}
\item  {$\frac{e+1}{\sqrt{e}}$}
\item  {$\frac{e-1}{2\sqrt{e}}$}
\item  {$\frac{e+1}{2\sqrt{e}}$}
\end{enumerate}

\item {Let $a_1, a_2, a_3 \cdots$ be terms on A.P. If $\frac{a_1+a_2+\cdots a_p}{a_1+a_2+\cdots a_q}= \frac{p^2}{q^2},p \neq q$, then $\frac{a_6}{a_{21}}$ equals}\\ \textcolor {magenta}{\hfill{[2006]}} \\
\begin{enumerate}[label={(\alph*)}]
\item  {$\frac{41}{11}$} \item  {$\frac{7}{2}$}
\item  {$\frac{2}{7}$} \item  {$\frac{11}{41}$}
\end{enumerate}




\end{enumerate}

\end{document}

\let\negmedspace\undefined
\let\negthickspace\undefined
\documentclass[journal]{IEEEtran}
\usepackage[a5paper, margin=10mm, onecolumn]{geometry}
%\usepackage{lmodern} % Ensure lmodern is loaded for pdflatex
\usepackage{tfrupee} % Include tfrupee package

\setlength{\headheight}{1cm} % Set the height of the header box
\setlength{\headsep}{0mm}     % Set the distance between the header box and the top of the text

\usepackage{gvv-book}
\usepackage{gvv}
\usepackage{cite}
\usepackage{amsmath,amssymb,amsfonts,amsthm}
\usepackage{algorithmic}
\usepackage{graphicx}
\usepackage{textcomp}
\usepackage{xcolor}
\usepackage{txfonts}
\usepackage{listings}
\usepackage{enumitem}
\usepackage{mathtools}
\usepackage{gensymb}
\usepackage{comment}
\usepackage[breaklinks=true]{hyperref}
\usepackage{tkz-euclide} 
\usepackage{listings}
% \usepackage{gvv}                                        
\def\inputGnumericTable{}                                 
\usepackage[latin1]{inputenc}                                
\usepackage{color}                                            
\usepackage{array}                                            
\usepackage{longtable}                                       
\usepackage{calc}                                             
\usepackage{multirow}                                         
\usepackage{hhline}                                           
\usepackage{ifthen}                                           
\usepackage{lscape}

\usepackage{multicol}

% Marks the beginning of the document
\begin{document}
\bibliographystyle{IEEEtran}
\vspace{3cm}

\title{Assignment 2 - EE1030}
\author{ee24btech11018 - D. Swaraj Sharma}

% \maketitle
% \newpage
% \bigskip
{\let\newpage\relax\maketitle}

\renewcommand{\thefigure}{\theenumi}
\renewcommand{\thetable}{\theenumi}
\setlength{\intextsep}{10pt}

\numberwithin{equation}{enumi}
\numberwithin{figure}{enumi}
\renewcommand{\thetable}{\theenumi}

\section{E - Subjective Problems}

\begin{enumerate}[label={\arabic*.}]

	\item A curve '$C$' passes through \brak{2,0} and the slope at \brak{x,y} as $\frac{\brak{x+1}^2+\brak{y-3}}{x+1}$. Find the equation of the curve. Find the area bounded by curve and $x$-axis in fourth quadrant.  
		
		\hfill{\brak{2004-4 Marks}}
	
\item If length of tangent at any point on the curve $y=f$\brak{x} intercepted between the point and the $x$-axis in fourth quadrant. 
	\hfill{\brak{2005-4 Marks}}

\end{enumerate}

\section{F - Match the Following}

\begin{enumerate}

\item Match the statements/expressions in \textbf{Column I} with the open intervals in \textbf{Column II}.
	\hfill{\brak{2009}}

	\begin{minipage}[t]{0.5\textwidth}
		\textbf{Column I}
		\begin{enumerate}[label=\brak{\Alph*}]
	\item Interval contained on the domain of definition of non-zero solutions of the differential equation \brak{x-3}$^2+y^{\prime}+y=0$
	\item Interval containing the value of the integral \\$\int_1^5 \brak{x-1}\brak{x-2}\brak{x-3}\brak{x-4}\brak{x-5}dx$
	\item Interval in which at least one of the points of local maximum of $\cos^2 x +\sin x$ lies
	\item Interval in which $\tan^{-1}\brak{\sin x + \cos x}$ is increasing 
	\end{enumerate}
	\end{minipage}
		\begin{minipage}[t]{0.5\textwidth}
			\textbf{Column II}
			\begin{enumerate}[label=\brak{\alph*}]\addtocounter{enumi}{15}
			\item \brak{-\frac{\pi}{2}, \frac{\pi}{2}} \\
			\item \brak{0, \frac{\pi}{2}} \\
			\item \brak{\frac{\pi}{8}, \frac{5\pi}{4}} \\
			\item \brak{0, \frac{\pi}{8}} \\
			\item \brak{-\pi, \pi} \\
	\end{enumerate}
	\end{minipage}
\end{enumerate}

\section{H - Assertion \& Reason Type Questions}

\begin{enumerate}
\item Let solution $y=y$\brak{x} of the differential equation $x\sqrt{x^2-1}dy-y\sqrt{y^2-1}dx=0$ satisfy $y$\brak{2}$=\frac{2}{\sqrt3}$.

	\textbf{STATEMENT-1}: $y$\brak{x}$=\sec{\brak{\sec^{-1}{x}-\frac{\pi}{6}}}$ and 

\textbf{STATEMENT-2}: $y$\brak{x} is given by $\frac{1}{y}=\frac{2\sqrt3}{x}-\sqrt{1-\frac{1}{x^2}}$ 
\hfill{\brak{2008}}
\begin{enumerate}
\item STATEMENT-1 is True, STATEMENT-2 is True; STATEMENT-2 is a correct explanation for STATEMENT-1
\item STATEMENT-1 is True, STATEMENT-2 is True; STATEMENT-2 \textbf{is NOT} a correct explanation for STATEMENT-1
\item STATEMENT-1 is True, STATEMENT-2 is False
\item STATEMENT-2 is False, STATEMENT-2 is True
\end{enumerate}
\end{enumerate}

\section{I - Integer Value Correct Type}

\begin{enumerate}
	\item Let $y^{\prime}$\brak{x}$+y$\brak{x}$g^{\prime}$\brak{x}$=g\brak{x}$,$g^{\prime}$\brak{x},$y$\brak{0}$=0$, $x\epsilon\mathbb{R}$, where $f^{\prime}$\brak{x} denotes $\frac{{df\brak{x}}}{dx}$ and $g$\brak{x} is a given non-constant differentiable function on $\mathbb{R}$ with $g$\brak{0}$=g$\brak{2}$=0$. Then the value of $y$\brak{2} is 
		\hfill{\brak{2011}}

\item Let $f:\mathbb{R}\to\mathbb{R}$ be a differentiable function with $f$\brak{0}$=0$. If $y=f$\brak{x} satisfies the differential equation $\frac{dy}{dx}=$\brak{2+5y}$+$\brak{5y-2}, then the value of $\lim\limits_{x\to-\infty}f$\brak{x} is. 

	\hfill{\brak{JEE Adv. 2018}}

	\item Let $f:\mathbb{R}\to\mathbb{R}$ be a differentiable function with $f$\brak{0}$=1$ and satisfying the differential equation $f$\brak{x+y}$=f$\brak{x}$f^{\prime}$\brak{y}$+f^{\prime}$\brak{y}$f$\brak{x} for all $x$,$y$ $\epsilon \mathbb{R}$ then, the value of $\log_e$ \brak{f\brak{4}} is. 
		\hfill{\brak{JEE Adv. 2018}}

\end{enumerate}

\section{Section-B // JEE Main / AIEEE}

\begin{enumerate}

\item The order and degree of the differential equation \brak{1+3\frac{dy}{dx}}$^{\frac{2}{3}} = 4\frac{d^3y}{dx^3}$ are 
	\hfill{\sbrak{2002}}

	\begin{enumerate}
			\begin{multicols}{2}
	\item \brak{1, \frac{2}{3}}
	\item \brak{3, 1}
	\item \brak{3, 3}
	\item \brak{1, 2}
	\end{multicols}
	\end{enumerate}
		
\item The solution of the equation $\frac{d^2y}{dx^2}=e^{-2x}$ 
	\hfill{\sbrak{2002}}

	\begin{enumerate}
			\begin{multicols}{2}
\item $\frac{e^{-2x}}{4}$
\item $\frac{e^{-2x}}{4}+cx+d$
\item $\frac{1}{4}e^{-2x}+cx^2+d$
\item $\frac{1}{4}e^{-4x}+cx+d$

	\end{multicols}
	\end{enumerate}

\item The degree and order of the differential equation of the family of all parabolas whose axis $x$-axis, are respectively. 
	\hfill{\sbrak{2003}}

	\begin{enumerate}
			\begin{multicols}{2}
	\item $2, 3$
	\item $2, 1$
	\item $1, 2$
	\item $3, 2$
	\end{multicols}
	\end{enumerate}

\item The solution of the differential equation \brak{1+y^2}$+$\brak{x-e^{\tan^{-1}{y}}}$\frac{dy}{dx}=0$, is 
	\hfill{\sbrak{2003}}
		\begin{enumerate}
				\begin{multicols}{2}
				\item $xe^{2\tan^{-1}y}=e^{\tan^{-1}}+k$
				\item $\brak{x-2}=ke^{2\tan^{-2}y}$
				\item $2xe^{\tan^{-1}y}=e^{2\tan^{-1}y}+k$
				\item $xe^{\tan^{-1}y}=\tan^{-2}y+k$
				\end{multicols}
		\end{enumerate}
	\item The differential equation for the family of circle $x^2+y^2-2ay=0$, where a is an arbitrary contant is 
		\hfill{\sbrak{2004}}
		\begin{enumerate}
				\begin{multicols}{2}
				\item $\brak{x^2+y^2}y^{/prime}=2xy$
				\item $2\brak{x^2+y^2}y^{/prime}=xy$
				\item $\brak{x^2-y^2}y^{/prime}=2xy$
				\item $2\brak{x^2-y^2}y^{/prime}=xy$
				\end{multicols}
		\end{enumerate}

	\item Solution of the differentisl equation $ydx+\brak{x+x^2y}dy=0$ is 
		\hfill{\sbrak{2004}}
		\begin{enumerate}
				\begin{multicols}{2}
				\item $\log y =Cx$
				\item $-\frac{1}{xy}+\log y=C$
				\item $\frac{1}{xy}+\log y=C$
				\item $-\frac{1}{xy}=C$
				\end{multicols}
		\end{enumerate}

	\item The differential equation representing the family of curves $y^2=2c\brak{x+\sqrt{c}}$, where $c>0$, is a parameter, is of order and degree as follows: 
		\hfill{\sbrak{2005}}
		\begin{enumerate}
				\begin{multicols}{2}
				\item {order $1$, degree $2$}
				\item {order $1$, degree $1$}
				\item {order $1$, degree $3$}
				\item {order $2$, degree $2$}
				\end{multicols}
		\end{enumerate}

	\item If $x\frac{dy}{dx}=y\brak{\log y - \log x +1}$, then the solution of the equation is 
		\hfill{\sbrak{2005}}
		\begin{enumerate}
				\begin{multicols}{2}
				\item $y \log \brak{\frac{x}{y}} =cx$
				\item $y \log \brak{\frac{y}{x}} =cy$
				\item $\log \brak{\frac{y}{x}} =cx$
				\item $\log \brak{\frac{x}{y}} =cy$
				\end{multicols}
		\end{enumerate}
\end{enumerate}

\end{document}

\let\negmedspace\undefined
\let\negthickspace\undefined
\documentclass[journal]{IEEEtran}
\usepackage[a5paper, margin=10mm, onecolumn]{geometry}
%\usepackage{lmodern} % Ensure lmodern is loaded for pdflatex
\usepackage{tfrupee} % Include tfrupee package

\setlength{\headheight}{1cm} % Set the height of the header box
\setlength{\headsep}{0mm}     % Set the distance between the header box and the top of the text

\usepackage{gvv-book}
\usepackage{gvv}
\usepackage{cite}
\usepackage{amsmath,amssymb,amsfonts,amsthm}
\usepackage{algorithmic}
\usepackage{graphicx}
\usepackage{textcomp}
\usepackage{xcolor}
\usepackage{txfonts}
\usepackage{listings}
\usepackage{enumitem}
\usepackage{mathtools}
\usepackage{gensymb}
\usepackage{comment}
\usepackage[breaklinks=true]{hyperref}
\usepackage{tkz-euclide} 
\usepackage{listings}
% \usepackage{gvv}                                        
\def\inputGnumericTable{}                                 
\usepackage[latin1]{inputenc}                                
\usepackage{color}                                            
\usepackage{array}                                            
\usepackage{longtable}                                       
\usepackage{calc}                                             
\usepackage{multirow}                                         
\usepackage{hhline}                                           
\usepackage{ifthen}                                           
\usepackage{lscape}

\usepackage{multicol}

% Marks the beginning of the document
\begin{document}
\bibliographystyle{IEEEtran}
\vspace{3cm}

\title{Assignment 2 - EE1030}
\author{ee24btech11018 - D. Swaraj Sharma}

% \maketitle
% \newpage
% \bigskip
{\let\newpage\relax\maketitle}
\renewcommand{\thefigure}{\theenumi}
\renewcommand{\thetable}{\theenumi}
\setlength{\intextsep}{10pt}
\numberwithin{equation}{enumi}
\numberwithin{figure}{enumi}
\renewcommand{\thetable}{\theenumi}
\section{Section - A}
\begin{enumerate}
\item Let $M$ denote the median of the following frequency distributions.\\
	\begin{table}[h!]
		\centering
	\begin{tabular}{|c|c|c|c|c|c|}
\hline
Class & 0-4 & 4-8 & 8-12 & 12-16 & 16-20 \\ 
\hline
Frequency & 3 & 9 & 10 & 8 & 6 \\
\hline
\end{tabular}	
	\end{table}\\
	Then $20 M$ is equal to:
		\begin{enumerate}
	\item $416$
	\item $104$
	\item $52$
	\item $208$
		\end{enumerate}
	\item If $f\brak{x} = \mydet{2\cos^4x & 2\sin^4x & 3+\sin^2x \\ 3+2\cos^4x & 2\sin^4x & \sin^22x \\ 2\cos^4x & 3+\sin^4x & \sin^22x}$ then $\frac{1}{5}f^{\prime}\brak{0}$ is equal to 
		\begin{enumerate}
			\item $0$
			\item $2$
			\item $2$
			\item $6$
		\end{enumerate}
	\item Let $A\brak{2, 3, 5}$ and $C\brak{-3, 4, -2}$ be opposite vertices of a parallelogram $ABCD$. If the diagonal $\overrightarrow{BD} = \hat{i}+2\hat{j}+3\hat{k}$ then the area of the parallelogram is equal to
		\begin{enumerate}
			\item $\frac{1}{2}\sqrt{410}$
			\item $\frac{1}{2}\sqrt{474}$
			\item $\frac{1}{2}\sqrt{586}$
			\item $\frac{1}{2}\sqrt{306}$
		\end{enumerate}
	\item If $2\sin^3x + \sin 2x \cos x + 4\sin x - 4 = 0$ has exactly $3$ solutions in the interval $\sbrak{0,\frac{n\pi}{2}}, n \in \mathbb{N}$, then the roots of the equation $x^2+nx+\brak{n-3}=0$ belong to:
		\begin{enumerate}
			\item $\brak{0, \infty}$
			\item $\brak{-\infty, 0}$
			\item $\brak{-\frac{\sqrt{17}}{2}, \frac{\sqrt{17}}{2}}$
			\item $\mathbb{Z}$
		\end{enumerate}
	\item Let $f:\sbrak{-\frac{\pi}{2}, \frac{\pi}{2}} \to \mathbb{R}$ be a differentiable function such that $f\brak{0}=\frac{1}{2}$. If the $\lim\limits_{x\to0} \frac{x\int_o^xf\brak{t}dt}{e^{x^2}-1}=\alpha$, then $8\alpha^2$ is equal to:
		\begin{enumerate}
			\item $16$
			\item $2$
			\item $1$
			\item $4$
		\end{enumerate}
\end{enumerate}
\section{Section - B}
\begin{enumerate}
	\item A group of $40$ students appeared in an examination of $3$ subjects - Mathematics, Physics \& Chemistry. It was found that all students passed in at least one of the subjects, $20$ students passed in Mathematics, $25$ students passed in Physics, $16$ students passed in Chemistry, at most $11$ students passed in both Mathematics and Physics, at most $15$ students passed in both Physics and Chemistry, at most $15$ students passed in both Mathematics and Chemistry. The maximum number of students passed in all the three subjects is 
	\item If $d_1$ is the shortest distance between the lines $x + 1 = 2y = -12z$, $x = y + 2 = 6z - 6$ and $d_2$ is the shortest distance between the lines $\frac{x-1}{2}=\frac{y+8}{-7}=\frac{z-4}{5}$, $\frac{x-1}{2}=\frac{y-2}{1}=\frac{z-6}{-3}$, then the value of $\frac{32\sqrt{3d_1}}{d_2}$ is :
	\item Let the latus rectum of the hyperbola $\frac{x^2}{9}-\frac{y^2}{b^2}=1$ subtend and angle of $\frac{\pi}{3}$ at the centre of the hyperbola. If $b^2$ is equal to $\frac{l}{m}\brak{1+\sqrt{n}}$, where $l$ and $m$ are the co-prime numbers, then $l^2+m^2+n^2$ is equal to
	\item Let $A = \cbrak{1, 2, 3, \dots 7}$ and let $P\brak{A}$ denote the power set of $A$. If the number of functions $f:A\to P\brak{A}$ such that $a\in f\brak{a}$,$\forall a\in A$ is $m^n$, $m$ and $n\in\mathbb{N}$ and $m$ is least, then $m+n$ is equal to
	\item The value $9\int_0^9\sbrak{\sqrt{\frac{10x}{x+1}}}dx$, where $\sbrak{t}$ denotes the greatest integer less than or equal to $t$, is 
	\item Number of integral terms in the expansion of $\cbrak{7^{\brak{\frac{1}{2}}}+11^{\brak{\frac{1}{6}}}}^{824}$ is equal to
	\item Let $y = y\brak{x}$ be the solution of the differential equation $\brak{1-x^2}dy=\sbrak{xy+\brak{x^3+2}\sqrt{3\brak{1-x^2}}}dx$, $-1<x<1$, $y\brak{0}=0$. If $y\brak{\frac{1}{2}}=\frac{m}{n}$, $m$ and $n$ are co-prime numbers, then $m+n$ is equal to 
	\item Let $\alpha$, $\beta \in \mathbb{N}$ be the roots of the equation $x^2-70x+\lambda=0$, where $\frac{\lambda}{2},\frac{\lambda}{3}\notin \mathbb{N}$. if $\lambda$ assumes the minimum possible value, then $\frac{\brak{\sqrt{\alpha-1}+\sqrt{\beta-1}}\brak{\lambda+35}}{\abs{\alpha-\beta}}$ is equal to:
	\item If the function $f\brak{x} = \begin{cases} \frac{1}{\abs{x}},& \abs{x}\geq2\\ ax^2+2b,& \abs{x}<2\end{cases}$ is differentiable on $\mathbb{R}$, then $48\brak{a+b}$ is equal to 
	\item Let $\alpha=1^2+4^2+8^2+13^2+19^2+26^2+\cdots$ upto $10$ terms and $\beta=\sum\limits_{n=1}^{10}n^4$. If $4\alpha-\beta=55k+40$, then $k$ is equal to 
\end{enumerate}

\end{document}

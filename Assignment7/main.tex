\let\negmedspace\undefined
\let\negthickspace\undefined
\documentclass[journal]{IEEEtran}
\usepackage[a5paper, margin=10mm, onecolumn]{geometry}
%\usepackage{lmodern} % Ensure lmodern is loaded for pdflatex
\usepackage{tfrupee} % Include tfrupee package

\setlength{\headheight}{1cm} % Set the height of the header box
\setlength{\headsep}{0mm}     % Set the distance between the header box and the top of the text

\usepackage{gvv-book}
\usepackage{gvv}
\usepackage{cite}
\usepackage{amsmath,amssymb,amsfonts,amsthm}
\usepackage{algorithmic}
\usepackage{graphicx}
\usepackage{textcomp}
\usepackage{xcolor}
\usepackage{txfonts}
\usepackage{listings}
\usepackage{enumitem}
\usepackage{mathtools}
\usepackage{gensymb}
\usepackage{comment}
\usepackage[breaklinks=true]{hyperref}
\usepackage{tkz-euclide} 
\usepackage{listings}
% \usepackage{gvv}                                        
\def\inputGnumericTable{}                                 
\usepackage[latin1]{inputenc}                                
\usepackage{color}                                            
\usepackage{array}                                            
\usepackage{longtable}                                       
\usepackage{calc}                                             
\usepackage{multirow}                                         
\usepackage{hhline}                                           
\usepackage{ifthen}                                           
\usepackage{lscape}

\usepackage{multicol}

% Marks the beginning of the document
\begin{document}
\bibliographystyle{IEEEtran}
\vspace{3cm}

\title{Assignment 7 - EE1030}
\author{ee24btech11018 - D. Swaraj Sharma}

% \maketitle
% \newpage
% \bigskip
{\let\newpage\relax\maketitle}
\renewcommand{\thefigure}{\theenumi}
\renewcommand{\thetable}{\theenumi}
\setlength{\intextsep}{10pt}
\numberwithin{equation}{enumi}
\numberwithin{figure}{enumi}
\renewcommand{\thetable}{\theenumi}
\begin{enumerate}
	\item Combustion between fuel \brak{octane} and oxidizer \brak{air} occurs inside a combustor with the following stoichiometric chemical reaction:
			$2C_8H_{18}+\brak{25O_2+94N_2}\to16CO_2+18H_2O+94N_2$
		The atomic weights of carbon $\brak{C}$, hydrogen $\brak{H}$, oxygen $\brak{O}$, and nitrogen $\brak{N}$ are $12$, $1$, $16$, and $14$, respectively. If the combustion takes place with the fuel to air ratio $0.028$, then the equivalence ratio of the fuel-oxidizer mixture is
		\begin{enumerate}
			\item $0.094$
			\item $0.422$
			\item $0.721$
			\item $2.371$
		\end{enumerate}
	\item The von Mises yield criterion or the maximum distrotion energy criterion for a plane stress problem with $\sigma_1$ and $\sigma_2$ as the principal stresses in the plane, and $\sigma_3$ as the yield stress, requires
		\begin{enumerate}
			\item $\sigma_1^2-\sigma_1\sigma_2+\sigma_2^2\leq\sigma_3^2$
			\item $\abs{\sigma_1-\sigma_2} \leq \sigma_3$
			\item $\abs{\sigma_1}\leq\sigma_3$
			\item $\abs{\sigma_2}\leq\sigma_3$
		\end{enumerate}
	\item An Euler-Bernoulli beam having a rectangular cross-section, as shown in the figure, is subjected to a non-uniform bending moment along its length. $V_z = \frac{dM_y}{dx}$. The shear stress distribution $\tau_{xz}$ across its cross-section is given by
		\begin{enumerate}
			\item $\tau_{xz} = \frac{V_z}{2l_y}\frac{z}{\brak{h/2}}$
			\item $\tau_{xz} = \frac{V_z\brak{h/2}^2}{2l_y}\brak{1-\frac{z^2}{\brak{h/2}^2}}$
			\item $\tau_{xz} = \frac{V_z}{2l_y}\brak{\frac{z}{\brak{h/2}}}^2$
			\item $\tau_{xz} = \frac{V_z\brak{h/2}^2}{2l_y}$
		\end{enumerate}
	\item At a stationary point of a multi-variable function, which of the following is true?
		\begin{enumerate}
			\item Curl of the function becomes unity
			\item Gradient of the function vanishes
			\item Divergence of the function vanishes
			\item Gradient of the function is maximum
		\end{enumerate}
	\item In a rocket engine, the hot gas generated in the combustion chamber exits the nozzle with a mass flow rate $719 kg/s$ and velocity $1794 m/s$. The area of the nozzle exit section is $0.635 m^2$. If the nozzle expansion is optimum, then the thrust produced by the engine is 
		\begin{enumerate}
			\item $811 kN$
			\item $1290 kN$
			\item $1354 kN$
			\item $2172 kN$
		\end{enumerate}
	\item For the control volume shown in the figure below, the velocities are measured both at the upstream and the downstream ends.
		\begin{figure}
			
\centering
\resizebox{0.5\textwidth}{!}{%
\begin{circuitikz}
\tikzstyle{every node}=[font=\huge]
\draw  (2.5,13.75) rectangle (6,7.25);
\draw [<->, >=Stealth] (2.5,6.75) -- (6,6.75);
\draw [<->, >=Stealth] (1.25,13.75) -- (1.25,7.25);
\draw (1,13.75) to[short] (1.5,13.75);
\draw (1,7.25) to[short] (1.5,7.25);
\draw (2.5,7) to[short] (2.5,6.5);
\draw (6,7) to[short] (6,6.5);
\draw [->, >=Stealth] (4,10.5) -- (8.5,10.5);
\draw [->, >=Stealth] (4,10.5) -- (4,15.5);
\node [font=\LARGE] at (0.5,10.5) {};
\node [font=\LARGE] at (3.5,15.5) {z};
\node [font=\LARGE] at (0.25,10.5) {h};
\node [font=\LARGE] at (4,6) {b};
\node [font=\LARGE] at (8.5,10) {y};
\node [font=\huge] at (6.5,15) {$V_z$};
\draw (4.5,15.5) to[short] (4,15);
\draw (4.5,15.5) to[short] (5,15);
\draw (4,15) to[short] (4.5,15);
\draw (4.25,15) to[short] (4.25,12.5);
\draw (4.5,15) to[short] (5,15);
\draw (4.75,15) to[short] (4.75,12.5);
\draw (4.25,12.5) to[short] (4.75,12.5);
\end{circuitikz}
}%




		\end{figure}
		The flow of density $\rho$ is incompressible, two dimensional and steady. The pressure is $p_{\infty}$ over the entire surface of the control volume. The drag on the airfoil is given by,
		\begin{enumerate}
			\item $\frac{\rho U^2_{\infty}h}{3}$
			\item $0$
			\item $\frac{\rho U^2_{\infty}h}{6}$
			\item $2\rho U^2_{\infty}h$
		\end{enumerate}
	\item A gas turbine engine operates with a constant area duct combustor with inlet and outlet stagnation temperatures $540 K$ and $1104 K$ respectively. Assume that the flow is one dimensional, incompressible and frictionless and that the heat addition is driving the flow inside the combustor. The pressure loss factor \brak{stagnation pressure loss non-dimensionalized by the inlet dynamic pressure} of the combustor is
		\begin{enumerate}
			\item $ 0$
			\item $0.489$
			\item $1.044$
			\item $2.044$
		\end{enumerate}
	\item The diffuser of an airplane engine decelerates the airflow from the flight Mach number $0.85$ to compressor inlet Mach number $0.38$. Assume that the ratio of the specific heats is constant and equal to $1.4$. If the diffuser pressure recovery ratio is $0.92$, then the isentropic efficiency of the diffuser is
		\begin{enumerate}
			\item $0.631$
			\item $0.814$
			\item $0.892$
			\item $1.343$
		\end{enumerate}
	\item An airfoil section is known to generate lift when placed in a uniform stream of speed $U_{\infty}$ at an incident $\alpha$. A biplane consisting of two such sections of identical chord $c$, separated by a distance $h$ is shown in the following figure:
FIG
		With regard to this biplane, which of the following statements is true?
		\begin{enumerate}
			\item Both the airfoils experience an upwash and increased approach velocity.
			\item Both the airfoils experience a downwash and decreased approach velocity.
			\item Both the airfoils experience an upwash and airfoil A experiences a decreased approach velocity while airfoil B experiences an increased approach velocity.
			\item The incidence for the individual sections of the biplane are not altered
		\end{enumerate}
	\item Numerical value of the integral
		$J=\int_0^1\frac{1}{1-x^2}dx$, if evalutaed numerically using the Trapezoidal rule with $dx = 0.2$ would be
		\begin{enumerate}
			\item $1$
			\item $\pi/4$
			\item $0.7837$
			\item $0.2536$
		\end{enumerate}
	\item The purpose of a fuel injection system in the combustor is
		\begin{enumerate}
			\item to accelerate the flow in the combustor
			\item to increase the stagnation pressure of the fuel-air mixture
			\item to ignite the fuel-air mixture
			\item to convert the bulk fuel into tiny droplets
		\end{enumerate}
	\item Which one of the following values is nearer to the vaccum specific impluse of a rocket engine using liquid hydrogen and liquid oxygen as propellants?
		\begin{enumerate}
			\item $49 sec$
			\item $450 sec$
			\item $6000 sec$
			\item $40000 sec$
		\end{enumerate}
	\item A circular cylinder is placed in a uniform stream of ideal fluid with its axis normal to the flow. Relative to the forward stagnation point, the angular positions along the circumference where the speed along the surface of the cylinder is equal to the free stream speed are
		\begin{enumerate}
			\item 30, 150, 210 and 330 degrees
			\item 45, 135, 225 and 270 degrees
			\item 0, 90, 180 and 270 degrees
			\item 60, 120, 240 and 300 degrees
		\end{enumerate}
	\item The Newton-Raphson interation formula to find a cube root of a positive number $c$ is
		\begin{enumerate}
			\item $x_{k+1} = \frac{2x_k^3+\sqrt[3]{c}}{3x_k^2}$
			\item $x_{k+1} = \frac{2x_k^3-\sqrt[3]{c}}{-3x_k^2}$
			\item $x_{k+1} = \frac{2x_k^3+{c}}{3x_k^2}$
			\item $x_{k+1} = \frac{x_k^3+{c}}{3x_k^3}$
		\end{enumerate}
	\item The torsion constant $J$ of a thin-walled closed tube of thickness t and mean radius $r$ is given by
		\begin{enumerate}
			\item $J=2\pi rt^3$
			\item $J=2\pi r^3t$
			\item $J=2\pi r^2t^2$
			\item $J=2\pi r^4$
		\end{enumerate}
	\item An aerospace system shown in the following figure is designed in such a way that the expansion generated at A is completely absorbed by wall B for $p_l=p_d$, where $p_d$ corresponds to the design condition.
		FIG
		For $p_1>p_{\infty}$ which of the following statements is NOT true?
		\begin{enumerate}
			\item For $p_l<p_d$, the expansion fan from A gets reflected from B as a compression wave
			\item For $p_l>p_d$, the expansion fan from A gets reflected from B as an expansion wave
			\item For $p_l<p_d$, the expansion fan from A gets reflected from B as an expansion wave
			\item For $p_l>p_d$, B always sees an expansion
		\end{enumerate}
	\item The span-wise lift distribution for three wings is shown in the following figure:
		FIG
		In the above figure, $c_l$ refers to the section list coefficient, $C_L$ refers to the lift coefficient of the wimd, $y$ is the coordinate along the span and $s$ is the span of the wing. A designer prefers to use a wing for which the stall begins at the root. Which of the wings will he choose?
		\begin{enumerate}
			\item P
			\item Q
			\item R
			\item None
		\end{enumerate}
\end{enumerate}
\end{document}
